\begin{longtable}{|p{2cm}|p{3cm}|p{3cm}|p{3cm}|p{3cm}|}
\caption{Descripción de los casos de prueba del módulo Northwest} \label{tab:casos_prueba} \\
\hline
\textbf{TEST CASE ID} & \textbf{TEST DATE} & \textbf{TEST DESCRIPTION} & \textbf{TEST CONDITIONS} & \textbf{SEVERITY } \\ \hline
\endfirsthead
\hline
\textbf{STEP ID} & \textbf{STEP DESCRIPTION} & \textbf{TEST DATE} & \textbf{EXPECTED RESULTS} & \textbf{ACTUAL RESULTS} \\ \hline
\endhead
TC2-NW & 2024-10-17 & Verificar el comportamiento del sistema cuando la suma de la oferta y la demanda no es igual. & El editor debe estar configurado correctamente. Los nodos de origen y destino deben poder ser creados y conectados sin errores. Los pesos asignados deben ser valores numéricos válidos y compatibles con El sistema. La oferta total debe ser mayor que  la demanda. & MEDIA                                                                                                     \\ \\ \hline
\textbf{STEP ID} & \textbf{STEP DESCRIPTION} & \textbf{TEST DATE} & \textbf{EXPECTED RESULTS} & \textbf{ACTUAL RESULTS} \\ \hline
S1-2-NW & Ingresar 100 unidades de oferta para los nodos de origen & 2024-10-17 & El sistema debe aceptar la cantidad de oferta sin mostrar errores. & PASS. Se muestra la cantidad de ofertta sin mostrar errores. \\ \hline
S2-2-NW & Ingresar 80 unidades de demanda para los nodos de destino. & 2024-10-17 & El sistema debe aceptar la cantidad de demanda sin mostrar errores. & PASS. La cantidad de demanda se muestra sin errores. \\ \hline
S3-2-NW & Ejecutar el cálculo utilizando el método de la esquina noroeste. & 2024-10-17 & El sistema debe detectar que la oferta y la demanda no están balanceadas y ajustar automáticamente la solución (por ejemplo, añadiendo un nodo ficticio) para resolver el problema de transporte sin errores. & FAIL. El sistema no detecta que la oferta y la demanda no están balanceados, y no se ajusta la solución. \\ \hline
\end{longtable}