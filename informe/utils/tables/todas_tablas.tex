\small % Slightly reduce the font size for better fit
\renewcommand{\arraystretch}{1.0} % Slightly reduce the space between rows
\setlength{\tabcolsep}{4pt} % Reduce space between columns

\begin{longtable}{|p{2cm}|p{3cm}|p{3cm}|p{3cm}|p{3cm}|}
  \caption{Caso de prueba 1} \label{tab:casos_prueba1} \\
  \hline
  \multicolumn{5}{|l|}{\textbf{USER STORY REFERENCE: HU009-NorthWest-01, HU006-GraphEditor}} \\ \hline
  \textbf{TEST CASE ID} & \textbf{TEST DATE} & \textbf{TEST DESCRIPTION} & \textbf{TEST CONDITIONS} & \textbf{SEVERITY} \\ \hline
  \endfirsthead
  \hline
  \textbf{STEP ID} & \textbf{STEP DESCRIPTION} & \textbf{TEST DATE} & \textbf{EXPECTED RESULTS} & \textbf{ACTUAL RESULTS} \\ \hline
  \endhead
  TC1-NW & 2024-10-17 & Comprobar que, al crear los nodos y las conexiones, se genera automáticamente la matriz con los valores correctos de los pesos. & El editor debe estar configurado correctamente. Los nodos de origen y destino deben poder ser creados y conectados sin errores. Los pesos asignados deben ser valores numéricos válidos y compatibles con el sistema. & ALTA \\ \hline
  \textbf{STEP ID} & \textbf{STEP DESCRIPTION} & \textbf{TEST DATE} & \textbf{EXPECTED RESULTS} & \textbf{ACTUAL RESULTS} \\ \hline
  S1-NW & Crear 3 nodos de origen y 3 nodos de destino en el editor. & 2024-10-17 & Los 6 nodos son creados correctamente sin errores. & \\ \hline
  S2-NW & Asignar conexiones entre ellos con pesos específicos para cada enlace. & 2024-10-17 & Las conexiones se establecen correctamente y los pesos asignados son visibles en la interfaz. & \\ \hline
  S3-NW & Abrir el formulario "Northwest" para ver la matriz generada. & 2024-10-17 & La matriz generada muestra los valores de los pesos correctamente, coincidiendo con los valores de las conexiones creadas. & \\ \hline
\end{longtable}


\begin{longtable}{|p{2cm}|p{3cm}|p{3cm}|p{3cm}|p{3cm}|}
    \caption{Caso de prueba 2} \label{tab:casos_prueba2} \\
    \hline
    \multicolumn{5}{|l|}{\textbf{USER STORY REFERENCE: HU009-NorthWest-01, HU006-GraphEditor}} \\ \hline
    \textbf{TEST CASE ID} & \textbf{TEST DATE} & \textbf{TEST DESCRIPTION} & \textbf{TEST CONDITIONS} & \textbf{SEVERITY } \\ \hline
    \endfirsthead
    \hline
    \textbf{STEP ID} & \textbf{STEP DESCRIPTION} & \textbf{TEST DATE} & \textbf{EXPECTED RESULTS} & \textbf{ACTUAL RESULTS} \\ \hline
    \endhead
    TC2-NW & 2024-10-17 & Verificar el comportamiento del sistema cuando la suma de la oferta y la demanda no es igual. & El editor debe estar configurado correctamente. Los nodos de origen y destino deben poder ser creados y conectados sin errores. Los pesos asignados deben ser valores numéricos válidos y compatibles con El sistema. La oferta total debe ser mayor que  la demanda. & MEDIA                                                                                                     \\ \\ \hline
    \textbf{STEP ID} & \textbf{STEP DESCRIPTION} & \textbf{TEST DATE} & \textbf{EXPECTED RESULTS} & \textbf{ACTUAL RESULTS} \\ \hline
    S1-2-NW & Ingresar 100 unidades de oferta para los nodos de origen & 2024-10-17 & El sistema debe aceptar la cantidad de oferta sin mostrar errores. & PASS. Se muestra la cantidad de ofertta sin mostrar errores. \\ \hline
    S2-2-NW & Ingresar 80 unidades de demanda para los nodos de destino. & 2024-10-17 & El sistema debe aceptar la cantidad de demanda sin mostrar errores. & PASS. La cantidad de demanda se muestra sin errores. \\ \hline
    S3-2-NW & Ejecutar el cálculo utilizando el método de la esquina noroeste. & 2024-10-17 & El sistema debe detectar que la oferta y la demanda no están balanceadas y ajustar automáticamente la solución (por ejemplo, añadiendo un nodo ficticio) para resolver el problema de transporte sin errores. & FAIL. El sistema no detecta que la oferta y la demanda no están balanceados, y no se ajusta la solución. \\ \hline
\end{longtable}


\begin{longtable}{|p{2cm}|p{3cm}|p{3cm}|p{3cm}|p{3cm}|}
    \caption{Caso de prueba 3} \label{tab:casos_prueba3} \\
    \hline
    \multicolumn{5}{|l|}{\textbf{USER STORY REFERENCE: HU009-NorthWest-01, HU006-GraphEditor}} \\ \hline
    \textbf{TEST CASE ID} & \textbf{TEST DATE} & \textbf{TEST DESCRIPTION} & \textbf{TEST CONDITIONS} & \textbf{SEVERITY } \\ \hline
    \endfirsthead
    \hline
    \textbf{STEP ID} & \textbf{STEP DESCRIPTION} & \textbf{TEST DATE} & \textbf{EXPECTED RESULTS} & \textbf{ACTUAL RESULTS} \\ \hline
    \endhead
    TC3-NW & 2024-10-17 & Verificar que las opciones de maximizar y minimizar generen soluciones distintas. & El editor debe estar configurado correctamente, permitiendo crear y conectar nodos. Pesos numéricos válidos y matriz de costos conocida. & ALTA \\ \hline
    \textbf{STEP ID} & \textbf{STEP DESCRIPTION} & \textbf{TEST DATE} & \textbf{EXPECTED RESULTS} & \textbf{ACTUAL RESULTS} \\ \hline
    S1-3-NW & Configurar la matriz de costos. & 2024-10-17 & El sistema acepta la matriz de costos sin errores y la muestra correctamente. & PASS. Matriz aceptada y mostrada correctamente. \\ \hline
    S2-3-NW & Seleccionar maximización y calcular. & 2024-10-17 & El sistema calcula y muestra la solución sin errores. & PASS. Calcula solución sin errores pero no alerta sobre datos problemáticos. \\ \hline
    S3-3-NW & Verificar la solución en maximización. & 2024-10-17 & Solución visible y accesible en la interfaz. & PASS. Solución visible y accesible. \\ \hline
    S4-3-NW & Seleccionar minimización y calcular. & 2024-10-17 & El sistema calcula y muestra la solución sin errores. & PASS. Calcula correctamente en minimización. \\ \hline
    S5-3-NW & Comparar ambas soluciones. & 2024-10-17 & Las soluciones deben diferir según el criterio de optimización. & PASS. Solución máxima y mínima obtenidas. \\ \hline
\end{longtable}

\begin{longtable}{|p{2cm}|p{3cm}|p{3cm}|p{3cm}|p{3cm}|}
    \caption{Caso de prueba 4} \label{tab:casos_prueba4} \\
    \hline
    \multicolumn{5}{|l|}{\textbf{USER STORY REFERENCE: HU009-NorthWest-01, HU006-GraphEditor}} \\ \hline
    \textbf{TEST CASE ID} & \textbf{TEST DATE} & \textbf{TEST DESCRIPTION} & \textbf{TEST CONDITIONS} & \textbf{SEVERITY} \\ \hline

    \endfirsthead
    \hline
    \textbf{STEP ID} & \textbf{STEP DESCRIPTION} & \textbf{TEST DATE} & \textbf{EXPECTED RESULTS} & \textbf{ACTUAL RESULTS} \\ \hline
    \endhead
    TC4-NW & 2024-10-17 & Verificar que el sistema siempre genere una solución factible para diferentes configuraciones de oferta y demanda. & El editor debe estar configurado correctamente. Los nodos de origen y destino deben poder ser creados y conectados sin errores. Los pesos asignados deben ser valores numéricos válidos y compatibles con El sistema. Varias configuraciones de oferta y demanda, incluyendo configuraciones balanceadas y no balanceadas. & ALTA                                                                                           \\ \\ \hline
    \textbf{STEP ID} & \textbf{STEP DESCRIPTION} & \textbf{TEST DATE} & \textbf{EXPECTED RESULTS} & \textbf{ACTUAL RESULTS} \\ \hline
    S1-4-NW & Ingresar diferentes valores de oferta y demanda (balanceados y no balanceados). & 2024-10-17 & El sistema debe aceptar los valores de oferta y demanda sin errores, independientemente de si están balanceados. & PASS. Si acepta cualquier tipo de valores sin ningún error. \\ \hline
    S2-4-NW & Ejecutar el cálculo del problema de transporte utilizando el módulo "Northwest". & 2024-10-17 & El sistema debe generar una solución válida y consistente, sin importar si la configuración de oferta y demanda está balanceada. & FAIL. Si genera soluciones, aunque no siempre es valida \\ \hline
\end{longtable}

\begin{longtable}{|p{2cm}|p{3cm}|p{3cm}|p{3cm}|p{3cm}|}
    \caption{Caso de prueba 5} \label{tab:casos_prueba5} \\
    \hline
    \multicolumn{5}{|l|}{\textbf{USER STORY REFERENCE: HU009-NorthWest-01, HU006-GraphEditor}} \\ \hline
    \textbf{TEST CASE ID} & \textbf{TEST DATE} & \textbf{TEST DESCRIPTION} & \textbf{TEST CONDITIONS} & \textbf{SEVERITY} \\ \hline
    \endfirsthead
    \hline
    \textbf{STEP ID} & \textbf{STEP DESCRIPTION} & \textbf{TEST DATE} & \textbf{EXPECTED RESULTS} & \textbf{ACTUAL RESULTS} \\ \hline
    \endhead
    TC5-NW & 2024-10-17 & Asegurarse de que el formulario de entrada de datos sea intuitivo y que los campos obligatorios estén correctamente validados. & El editor debe estar configurado correctamente. Los nodos de origen y destino deben poder ser creados y conectados sin errores. Datos válidos e inválidos en el formulario de entrada, incluyendo campos vacíos y datos no numéricos en los campos de oferta y demanda. & ALTA \\ \hline
    \textbf{STEP ID} & \textbf{STEP DESCRIPTION} & \textbf{TEST DATE} & \textbf{EXPECTED RESULTS} & \textbf{ACTUAL RESULTS} \\ \hline
    S1-5-NW & Intentar enviar el formulario con algunos campos vacíos. & 2024-10-17 & El sistema debe mostrar mensajes de error específicos indicando que los campos obligatorios están vacíos. & FAIL. El sistema no muestra mensajes de error específicos ni indica que los campos son obligatorios. \\ \hline
    S2-5-NW & Ingresar datos no numéricos en los campos de oferta y demanda y enviar el formulario. & 2024-10-17 & El sistema debe mostrar mensajes de error indicando que los valores en oferta y demanda deben ser numéricos. & FAIL. El sistema no muestra mensajes de error ni indica que los valores necesiten ser numéricos. \\ \hline
    S3-5-NW & Ingresar datos válidos en todos los campos y enviar el formulario, sin demanda y oferta. & 2024-10-17 & El sistema debe lanzar un mensaje de error porque los datos de demanda y oferta deberían ser obligatorios. & FAIL. El sistema permite enviar los datos, pero no muestra error cuando los datos obligatorios de demanda y oferta están vacíos. \\ \hline
\end{longtable}

\small % Slightly reduce the font size for better fit
\renewcommand{\arraystretch}{1.0} % Slightly reduce the space between rows
\setlength{\tabcolsep}{4pt} % Reduce space between columns

\begin{longtable}{|p{2cm}|p{3cm}|p{3cm}|p{3cm}|p{3cm}|}
    \caption{Caso de prueba 6} \label{tab:casos_prueba6} \\
    \hline
    \multicolumn{5}{|l|}{\textbf{USER STORY REFERENCE: HU009-NorthWest-01, HU006-GraphEditor}} \\ \hline
    \textbf{TEST CASE ID} & \textbf{TEST DATE} & \textbf{TEST DESCRIPTION} & \textbf{TEST CONDITIONS} & \textbf{SEVERITY} \\ \hline
    \endfirsthead
    \hline
    \textbf{STEP ID} & \textbf{STEP DESCRIPTION} & \textbf{TEST DATE} & \textbf{EXPECTED RESULTS} & \textbf{ACTUAL RESULTS} \\ \hline
    \endhead
    TC6-NW & 2024-10-17 & Verificar que el editor permita a los usuarios definir nodos y conexiones con pesos, y que los grafos se visualicen de forma intuitiva en la interfaz. & Configuración inicial del editor sin grafos; usuarios con capacidad de definir nodos y conexiones. & ALTA \\ \hline
    \textbf{STEP ID} & \textbf{STEP DESCRIPTION} & \textbf{TEST DATE} & \textbf{EXPECTED RESULTS} & \textbf{ACTUAL RESULTS} \\ \hline
    S1-6-NW & Definir un nodo en el editor. & 2024-10-17 & El nodo debe aparecer inmediatamente en la visualización gráfica del editor. & PASS. El nodo aparece de manera inmediata en la visualización en el editor. \\ \hline
    S2-6-NW & Ingresar datos válidos en todos los campos y enviar el formulario. & 2024-10-17 & La conexión debe visualizarse en el grafo en tiempo real, mostrando el peso asignado. & PASS. La conexión se visualiza en tiempo real y asigna un peso 0. \\ \hline
    S3-6-NW & Intentar crear una conexión con datos inconsistentes (por ejemplo, un peso no numérico). & 2024-10-17 & El sistema debe notificar al usuario sobre la inconsistencia en la entrada de datos, previniendo errores en la creación del grafo. & FAIL. El sistema evita la inconsistencia de los datos, pero no notifica sobre errores. \\ \hline
    S4-6-NW & Modificar la conexión o el nodo existente. & 2024-10-17 & La visualización del grafo debe actualizarse de inmediato para reflejar los cambios realizados. & PASS. Los cambios al grafo se producen de manera inmediata. \\ \hline
\end{longtable}

\small % Slightly reduce the font size for better fit
\renewcommand{\arraystretch}{1.0} % Slightly reduce the space between rows
\setlength{\tabcolsep}{4pt} % Reduce space between columns

\begin{longtable}{|p{2cm}|p{3cm}|p{3cm}|p{3cm}|p{3cm}|}
    \caption{Caso de prueba 7} \label{tab:casos_prueba7} \\
    \hline
    \multicolumn{5}{|l|}{\textbf{USER STORY REFERENCE: HU009-NorthWest-01, HU006-GraphEditor}} \\ \hline
    \textbf{TEST CASE ID} & \textbf{TEST DATE} & \textbf{TEST DESCRIPTION} & \textbf{TEST CONDITIONS} & \textbf{SEVERITY} \\ \hline
    \endfirsthead
    \hline
    \textbf{STEP ID} & \textbf{STEP DESCRIPTION} & \textbf{TEST DATE} & \textbf{EXPECTED RESULTS} & \textbf{ACTUAL RESULTS} \\ \hline
    \endhead
    TC7-NW & 2024-10-17 & Verificar que el editor permita cargar grafos desde el computador en formato JSON y descargar los grafos creados, con notificaciones de éxito o error en las operaciones. & Grafo guardado en formato JSON en el computador; editor configurado para cargar y descargar grafos. & ALTA \\ \hline
    \textbf{STEP ID} & \textbf{STEP DESCRIPTION} & \textbf{TEST DATE} & \textbf{EXPECTED RESULTS} & \textbf{ACTUAL RESULTS} \\ \hline
    S1-7-NW & Cargar un archivo JSON de grafo desde el computador al editor. & 2024-10-17 & El grafo debe aparecer correctamente en el editor, conservando todos los nodos y conexiones, y el sistema debe notificar que la carga fue exitosa. & PASS. El grafo aparece correctamente en el editor, y conserva todos los nodos y conexiones. \\ \hline
    S2-7-NW & Intentar cargar un archivo JSON con datos incompletos o en formato incorrecto. & 2024-10-17 & El sistema debe mostrar un mensaje de error indicando el problema con el archivo, sin cargar datos incompletos en el editor. & FAIL. El sistema no muestra ningún mensaje de error. \\ \hline
    S3-7-NW & Descargar el grafo actualmente visualizado en el editor en formato JSON. & 2024-10-17 & El archivo debe descargarse correctamente, conservando toda la información de nodos y conexiones, y el sistema debe notificar que la descarga fue exitosa. & PASS. El archivo se descarga correctamente con toda la información, y el sistema notifica que la descarga fue exitosa. \\ \hline
\end{longtable}