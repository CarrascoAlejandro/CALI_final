\begin{longtable}{|p{2cm}|p{3cm}|p{3cm}|p{3cm}|p{3cm}|}
\caption{Descripción de los casos de prueba del módulo Northwest} \label{tab:casos_prueba} \\
\hline
\textbf{TEST CASE ID} & \textbf{TEST DATE} & \textbf{TEST DESCRIPTION} & \textbf{TEST CONDITIONS} & \textbf{SEVERITY } \\ \hline
\endfirsthead
\hline
\textbf{STEP ID} & \textbf{STEP DESCRIPTION} & \textbf{TEST DATE} & \textbf{EXPECTED RESULTS} & \textbf{ACTUAL RESULTS} \\ \hline
\endhead
TC1-NW & 2024-10-17 & Comprobar que, al crear los nodos y las conexiones, se genera automáticamente la matriz con los valores correctos de los pesos. & El editor debe estar configurado correctamente. Los nodos de origen y destino deben poder ser creados y conectados sin errores. Los pesos asignados deben ser valores numéricos válidos y compatibles con El sistema. & ALTA \\ \hline
\textbf{STEP ID} & \textbf{STEP DESCRIPTION} & \textbf{TEST DATE} & \textbf{EXPECTED RESULTS} & \textbf{ACTUAL RESULTS} \\ \hline
S1-1-NW & Crear 3 nodos de origen y 3 nodos de destino en el editor. & 2024-10-17 & Los 6 nodos son creados correctamente sin errores. & PASS. Los 6 nodos se crearon correctamente. \\ \hline
S2-1-NW & Asignar conexiones entre ellos con pesos específicos para cada enlace. & 2024-10-17 & Las conexiones se establecen correctamente y los pesos asignados son visibles en la interfaz. & PASS. Las conexiones inician con un peso 0, luego este valor es editable. \\ \hline
S3-1-NW & Abrir el formulario "Northwest" para ver la matriz generada. & 2024-10-17 & La matriz generada muestra los valores de los pesos correctamente, coincidiendo con los valores de las conexiones creadas. & PASS. Los valores de los pesos sean 0 o si se han editado corresponde correctamente \\ \hline
\end{longtable}