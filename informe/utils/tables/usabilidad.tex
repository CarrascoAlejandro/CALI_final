\begin{longtable}{|>{\raggedright\arraybackslash}p{10cm}|>{\centering\arraybackslash}p{3cm}|}
    \caption{Plantilla de Reporte de Usabilidad} \label{tab:reporte_usabilidad} \\
    \hline
    \textbf{Items} & \textbf{Evaluation} \\ \hline
    
    \textbf{1.- Visibilidad del estado del sistema} & \\ \hline
    ¿Cada parte de la interfaz comienza con un título que describa el contenido de la pantalla? & Conforme \\ \hline
    ¿El diseño de íconos y su estética es consistente en todo el sistema? & Conforme \\ \hline
    Cuando se selecciona un icono que está rodeado de otros iconos, ¿Se distingue claramente el ícono seleccionado? & No Conforme \\ \hline
    Si se utilizan ventanas emergentes (pop-up) para mostrar mensajes de error, ¿Permiten esas ventanas que el usuario visualice el error en la interfaz cuando se despliegan? & No Conforme\\ \hline
    ¿Hay algún tipo de feedback para cada acción u operación? & No Conforme\\ \hline
    Luego de que el usuario completa una acción o serie de acciones, ¿El "feedback" del sistema indica que el siguiente grupo de acciones puede completarse? & No Conforme\\ \hline
    El sistema provee algún tipo de feedback visual en menús o cajas de diálogo que indiquen qué opciones pueden seleccionarse. & Conforme \\ \hline
    El sistema provee algún tipo de feedback visual en menús o cajas de diálogo que indiquen en cuál de las posibles opciones se halla posicionado el cursor. & No Conforme \\ \hline
    Si hay menús o caja de diálogo en donde pueden seleccionarse múltiples opciones, ¿El sistema provee algún tipo de "feedback" visual que indique cuáles son las opciones ya seleccionadas? & No Conforme \\ \hline
    ¿El sitio web entrega información corporativa de la organización? & Conforme \\ \hline
    Si existen demoras mayores a 15 segundos en las respuestas del sistema, ¿El usuario es informado del progreso en la concreción de la respuesta? & No Conforme \\ \hline
    ¿Informa datos relevantes para quien no "navega" (Ej: Horas de atención)? ¿Y para hacer consultas web o no web (Ej: números de teléfono)? & No Conforme \\ \hline
    ¿Los tiempos de respuesta son apropiados para cada tarea? & Conforme \\ \hline
    Tiempo de escritura, movimiento del cursor o selección con el ratón: entre 0,5 y 1,5 milisegundos & Conforme \\ \hline
    Tareas más comunes: 2 a 4 segundos & Conforme \\ \hline
    Tareas complejas: 8 a 12 segundos & Conforme \\ \hline
    No son necesarios altos niveles de concentración y no es requerido retener información: 2 a 15 segundos & No Conforme\\ \hline
    La terminología usada en los menús, ¿Es consistente con el dominio de conocimiento del usuario en relación a la tarea a realizar? & Conforme \\ \hline
    ¿El usuario conoce su ruta de ubicación? & Conforme \\ \hline
    
    \textbf{2.- Relación entre el sistema y el mundo real} & \\ \hline
    ¿Los íconos son concretos y familiares para el usuario? & Conforme \\ \hline
    ¿Los colores seleccionados corresponden a los valores esperados? & No Conforme \\ \hline
    Cuando se ingresan datos en la pantalla, ¿La terminología utilizada para describir la tarea es familiar para los usuarios? & Conforme \\ \hline
    Cuando la pantalla incluye preguntas, ¿El lenguaje de esas preguntas es claro y conciso? & N/A \\ \hline
    Las combinaciones de secuencias de letras o palabras extrañas o poco frecuentes, ¿Se evitan siempre que sea posible? & No Conforme\\ \hline
    El sistema ingresa/elimina de manera automática los signos de pesos o dólar y decimal cuando se insertan valores monetarios. & N/A\\ \hline
    ¿Se utilizan nombres unívocos y descriptivos en todo momento? & No Conforme \\ \hline
    ¿Se hace uso de los rastreadores de progreso? & No Conforme\\ \hline
    Los H1 están optimizados para SEO & No Conforme \\ \hline
    
    \textbf{3.- Control y libertad  por parte del usuario} & \\ \hline
    En sistemas que permitan el uso de ventanas superpuestas ¿Es fácil reacomodar reubicar esas ventanas en la pantalla? & Conforme \\ \hline
    En sistemas que permitan el uso de ventanas superpuestas ¿Es fácil para los usuarios cambiar de una ventana a otra? & Conforme \\ \hline
    Cuándo una tarea efectuada por el usuario se completa ¿el sistema espera alguna señal del usuario antes de procesar la tarea? & Conforme \\ \hline
    ¿Se pregunta al usuario que confime acciones que tendrán consecuencias drásticas, negativas o destructivas? & No Conforme \\ \hline
    ¿Existe una función para "deshacer" al nivel de cada acción simple, cada entrada de datos y cada grupo de acciones completadas? & No Conforme \\ \hline
    ¿Los usuarios pueden cancelar acciones en progreso? & No Conforme \\ \hline
    ¿Los usuarios pueden reducir el tiempo de entrada de datos copiando y modificando datos existentes? & Conforme \\ \hline
    Los menús son anchos (muchos ítems), antes que profundos (muchos niveles) & No Conforme \\ \hline
    Si el sistema posee menús de niveles múltiples ¿Existe algún mecanismo que permita a los usuarios regresar al menú previo? & Conforme \\ \hline
    Los usuarios pueden moverse hacia delante o hacia atrás entre las opciones de campos o cajas de dialogo. & Conforme \\ \hline
    Si el sistema utiliza una interfaz de preguntas y respuestas ¿Pueden los usuarios regresar a la pregunta anterior o saltear hacia delante una pregunta? & N/A \\ \hline
    ¿Los usuarios pueden revertir sus acciones de manera sencilla? & Conforme \\ \hline
    Si el sistema permite a los usuarios revertir sus acciones , ¿Existe un mecanismo que permita "deshacer" varias acciones de manera simultánea?  & No Conforme \\ \hline

    \textbf{4.- Consistencia y estándares} & \\ \hline
    El abuso de letras en mayúscula en la pantalla se ha evitado & No Conforme \\ \hline
    No hay más de 12/20 tipos de íconos & Conforme \\ \hline
    Existe algún elemento visual que identifique la ventana activa & Conforme \\ \hline
    Cada ventana posee un título & Conforme \\ \hline
    ¿Es posible utilizar las barras de desplazamiento horizontal y vertical en cada ventana? & Conforme \\ \hline
    Si una opción de un menú es la de "salir" ¿Esta opción aparece como ultimo ítem en el menú? & No Conforme \\ \hline
    ¿Los títulos de los menús están centrados o justificados a la izquierda? & No Conforme\\ \hline
    Fuentes: hasta tres tipos como máximo & No Conforme\\ \hline
    Hasta cuatro colores (usados ocacionalmente)  & No Conforme \\ \hline
    Sonido: tonos suaves para dispositivos de retroalimentación ocacional y bruscos para condiciones críticas. & N/A\\ \hline
    ¿Se provee una leyenda si los códigos de color son numeros o dificiles de interpretar?  & No Conforme \\ \hline
    Se evitan los pares de colores espectralmente extremos y altamente  cromáticos & No Conforme \\ \hline
    Los azules saturados no se utilizan para texto u otro elemento pequeño. & No Conforme \\ \hline
    La información más importante esta above the fold (la parte del sitio que los usuarios ven primero) & No Conforme \\ \hline
    ¿La estructura de la entrada de datos es consistente entre las diferentes pantallas? & No Conforme \\ \hline

    \textbf{5.- Prevención de errores} & \\ \hline
    ¿Las entradas de datos no son sensibles a mayúsculas siempre que sea posible? & No Conforme \\ \hline
    Las pantallas para entrada de datos y cajas de diálogo indican el número de espacios en caracteres que estan disponibles para un campo & No Conforme \\ \hline
    Los campos en las pantallas de entrada de datos y las cajas de diálogo ¿contienen valores por defecto cuando corresponden? & No Conforme \\ \hline

    \textbf{6.- Reconocer antes que recordar} & \\ \hline
    ¿Las áreas de texto tienen "espacios de respiración" que las rodeen? & No Conforme \\ \hline
    ¿Se ha utilizado el mismo color para agrupar elementos relacionados? & No Conforme \\ \hline
    ¿Existe buen contraste de brillo y de color entre los colores usados para imágines y fondos? & No Conforme \\ \hline
    Los colores suaves, brillantes y saturados se han utilizado para enfatizar datos, mientras que los colores oscuros, opacos y no saturados, han sido usados para des-enfatizar datos? & No Conforme \\ \hline
    ¿Los ítems inactivos en un menú aprecen en gris o están omitidos? & No Conforme \\ \hline

    \textbf{7.- Flexibilidad y eficiencia en el uso} & \\ \hline
    Los usuarios pueden reducir el tiempo de entrada de datos si se les permite copiar y pegar datos existentes. & Conforme \\ \hline
    Si las listas de menú son cortas (siete ítem o menos) ¿Pueden los usuarios seleccionar un ítem moviendo el cursor? & Conforme \\ \hline

    \textbf{8.- Diseño estético y minimalista} & \\ \hline
    Los íconos son visuamente distinguibles de acuerdo a su significado conceptual  & Conforme \\ \hline
    ¿Cada ícono esta resaltado con respecto a su fondo? & Conforme \\ \hline
    Cada pantalla de entrada de datos incluye un título simple, corto, claro y suficientemente distintivo. & No Conforme \\ \hline
    Los títulos de los menús son breves pero lo suficientemente largos como para comunicar su contenido. & No Conforme \\ \hline

    \textbf{9.- Ayuda a los usuarios a reconocer, diagnosticar y recuperarse de los errores} & \\ \hline
    ¿Los sonidos son utilizados para señalar errores? & N/A \\ \hline
    Si se usan mensajes de error con humor ¿Son apropiados y respetuosos para la comunidad de usuarios? & No Conforme\\ \hline
    ¿Los mensajes de error son gramaticalmente correctos? & Conforme \\ \hline
    ¿Los mensajes de error evitan el uso de signos de admiración? & No Conforme\\ \hline
    Los mensajes de error evitan el uso de palabras violentas u hostiles & No Conforme\\ \hline
    Si se detecta un error en un campo de entrada de datos ¿El sistema posiciona el cursor en ese campo o lo resalta de alguna manera? & No Conforme \\ \hline
    ¿Los mensajes de error sugieren la causa del problema que lo has ha ocacionado? & No Conforme \\ \hline
    ¿Los mensajes de error indican que acción debe realizar el usuario para corregir el error correspondiente? & No Conforme \\ \hline

    \textbf{10.- Ayuda y documentación} & \\ \hline
    ¿Las instrucciones en linea se distnguen visualmente?  & No Conforme \\ \hline
    Si las opciones de los menús son ambiguas ¿el sistema provee información aclaratoria adacional cuando un ítem es seleccionado? & No Conforme \\ \hline
    ¿La función de ayuda del menú es visible? (Por ejemplo una tecla etiquetada AYUDA o un menú especial) & No Conforme \\ \hline
    Navegación: la información es facíl de encontrar & Conforme \\ \hline
    ¿La información es exacta, completa y comprensible? ¿La información es relevante? & No Conforme \\ \hline
    Tras haber accedido a la ayuda ¿Pueden los usuarios continuar con su trabajo desde donde ha sido interrumpido? & Conforme \\ \hline
    ¿Es fácil acceder y regresar del sistema de ayuda? & Conforme \\ \hline
\end{longtable}